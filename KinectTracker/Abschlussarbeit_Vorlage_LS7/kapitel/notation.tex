%========================================================================================
% TU Dortmund, Informatik Lehrstuhl VII
%========================================================================================

\chapter*{Mathematische Notation} \label{Notation}
\addcontentsline{toc}{chapter}{Mathematische Notation}

\newcommand{\tabdummy}{\midrule[0pt]}

\begin{tabular}{p{0.25\textwidth}p{0.65\textwidth}}
  \textbf{Notation} & \textbf{Bedeutung} \\ \toprule[1pt]
   $\mathbb{N}$ & Menge der natürlichen Zahlen ${1, 2, 3, \ldots}$ \\ \tabdummy
   $\mathbb{R}$ & Menge der reellen Zahlen \\ \tabdummy
   $\mathbb{R}^d$ & $d$-dimensionaler Raum\\ \tabdummy
   $\mathcal{M} = \{m_1,\ldots,m_N\}$ & Menge $\mathcal{M}$ von $N$ Elementen $m_i$ \\ \tabdummy
   $G=(V,E)$ & Graph $G$ mit Knotenmenge $V$ und Kantenmenge $E$ \\ \tabdummy
   $\mathbf{p}$ & Vektor\\ \tabdummy
   $\mathbf{p}_i$ & $i$-tes Element eines Vektors \\ \tabdummy
   $\mathbf{v}^{(j)}_i$ & $i$-tes Element des $j$-ten Vektors \\ \tabdummy
   $\mathbf{A}$ & Matrix
\end{tabular}
